\section{Introduction}

A recurring problem faced by developers working on large software projects is
that of making sure knowledge is accessible by every member of the team \cite{Ko:2007}.
Traditionally, teams have explored diverse methods of documentation, ranging
from inline comments in the source code to documents that provide an overview
of design and architecture.

However, documentation has a number of undesired properties that tend to make
it less attractive to a developer's workflow. Firstly, documentation needs to
be kept up to date with the underlying code. As software projects are
constantly evolving systems, this task can become challenging and is the
easiest to be neglected under the pressure of short deadlines. Secondly, since
documentation reflects the understanding of the developer who wrote it, it can
be misleanding, incomplete, or even hard to read. Finally, traditional
documentation artifacts cannot have its correctness automatically checked;
developers who find themselves unable to get the information they need from
documentation tend to resort to coworkers, making interruptions more frequent
and increasing dependency on a single individual \cite{Ko:2007}.

In this project, we propose \textbf{\texttt{doco}}, a tool that aims to improve
the productivity of a team working on a large software project by providing an
\emph{automated way to generate documentation.} When browsing large bodies of
source code, it is oftentimes important to know what are the implicit
expectations of functions: in what ranges its parameters are expected to be,
what program state is valid at that point, among others.  Most of these
properties can be deterministically inferred using source code analysis
techniques, such as static and dynamic analysis and symbolic execution. We
believe that by using \texttt{doco}, developers manipulating unfamiliar
codebases will be less likely to introduce changes that infringe implicit
assumptions that functions typically make.
