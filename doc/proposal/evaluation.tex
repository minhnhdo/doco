\section{Evaluation}

Since our tool aims to provide developers with accurate documentation about a program's source code in terms of pre and post-conditions and usage, the objective of our evaluation is twofold:\\
\newline
{\bf RQ 1: Does \texttt{doco} generate accurate documentation in terms of pre and post-conditions and usage examples from source code?}

To evaluate this goal, we will carry out an internal evaluation on a well-documented open-source Rust project called Itertools available on Github.\cite{bluss:2018} It has a well documented collection of usage of function examples along with their associated preconditions. To evaluate \texttt{doco}, we will generate documentation for itertools using \texttt{doco} and compare it with the existing one. We will report the accuracy of \texttt{doco} in generating correct conditions for usage, preconditions and post conditions separately.\newline

{\bf RQ 2: Does the \texttt{doco}-generated documentation help developers get up to speed with a new code base and implement functionality quickly and easily?}

We will present a group of 10 graduate students unfamiliar with the Rust-itertools library and ask them to implement 2 pre-defined set of trivial tasks of similar complexity using the library  (require use of same number of Itertools functions). The first set of tasks will be implemented by referring to source-code only while the other will be implemented using the \texttt{doco}-generated documentation. A comparison of the time it took for the developers to implement the two different set of tasks and the usefulness of the \texttt{doco}-generated documentation will be collected using a survey to evaluate the effectiveness of our tools in helping developers.
