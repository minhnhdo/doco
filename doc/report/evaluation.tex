\section{Evaluation}

In order to evaluate \texttt{doco}, we wanted to answer three main questions:

\begin{itemize}
    \item How often is \texttt{doco} able to provide documentation for Java methods?
    
    \item How accurate was the output generated by \texttt{doco}?
    
    \item How easy to understand is the documentation generated by \texttt{doco}?
\end{itemize}

In order to answer such questions, we generated documentation for a number of methods in the popular Guava \cite{guava:2018} library. The choice for an open-source project was based on the belief that it should be representative of large Java applications in general.

Guava ships with a comprehensive test suite and uses the JUnit library to implement unit tests. Unfortunately, Daikon does not support Java's reflection API, used by JUnit, and therefore it is very challenging to perform our dynamic analysis strategy on Guava. As an attempt to overcome this limitation, we manually re-wrote the test suite for one of the classes in the Guava project in order to make it work under Daikon's constraints (more specifically, the change involved creating a \texttt{main} function and invoking test methods explicitly). Unfortunately, Daikon's instrumentation adds a large overhead on the execution of the test suite and makes it impractical for projects like Guava. For these reasons, the numbers and data below refer only to the concolic execution module of \texttt{doco}, which does not depend on the size of the test suite.

We ran \texttt{doco}'s analysis on 98 methods defined in the Guava library, contained in 5 classes. The results of our evaluation are summarized in Table \ref{table:results}.

\begin{center}
\begin{table}[h]
    \begin{tabular}{ | l | l |}
    \hline
    Methods & Outcome \\ \hline
    54 & N \\ \hline  
    18 & I \\ \hline
    25 & V \\ \hline
    1 & W \\
    \hline
    \end{tabular}
    \caption{Results obtained when running \texttt{doco} on methods defined in the Guava library. Legend: N: no documentation is produced; I: an incomprehensible property is generated; V: valid and correct result reported by \texttt{doco}; W: a wrong property is inferred.}
    \label{table:results}
\end{table}
\end{center}

\texttt{doco} does not return any result (N category in Table~\ref{table:results}) for methods that require a reference parameter, e.g. an object like \texttt{Comparator}, or cause the JDart execution to diverge, or  use unsupported features in JDart like String. \texttt{doco} returns an incomprehensible result (I category in Table~\ref{table:results}) for pre-conditions that contain arithmetic or bitwise operations, which are currently not simplified. \texttt{doco} returns a wrong result for only one method, namely \texttt{com.google.common.math.IntMath::checkedMultiply}, due to the unfortunate values the JDart execution engine assigns to the parameters.

Most executions of \texttt{doco} return within two seconds, except for the executions for some methods for which \texttt{doco} returns no result.
